\documentclass{book}
\usepackage[a4paper,top=2.5cm,bottom=2.5cm,left=2.5cm,right=2.5cm]{geometry}
\usepackage{makeidx}
\usepackage{natbib}
\usepackage{graphicx}
\usepackage{multicol}
\usepackage{float}
\usepackage{listings}
\usepackage{color}
\usepackage{ifthen}
\usepackage[table]{xcolor}
\usepackage{textcomp}
\usepackage{alltt}
\usepackage{ifpdf}
\ifpdf
\usepackage[pdftex,
            pagebackref=true,
            colorlinks=true,
            linkcolor=blue,
            unicode
           ]{hyperref}
\else
\usepackage[ps2pdf,
            pagebackref=true,
            colorlinks=true,
            linkcolor=blue,
            unicode
           ]{hyperref}
\usepackage{pspicture}
\fi
\usepackage[utf8]{inputenc}
\usepackage{mathptmx}
\usepackage[scaled=.90]{helvet}
\usepackage{courier}
\usepackage{sectsty}
\usepackage{amssymb}
\usepackage[titles]{tocloft}
\usepackage{doxygen}
\lstset{language=C++,inputencoding=utf8,basicstyle=\footnotesize,breaklines=true,breakatwhitespace=true,tabsize=8,numbers=left }
\makeindex
\setcounter{tocdepth}{3}
\renewcommand{\footrulewidth}{0.4pt}
\renewcommand{\familydefault}{\sfdefault}
\hfuzz=15pt
\setlength{\emergencystretch}{15pt}
\hbadness=750
\tolerance=750
\begin{document}
\hypersetup{pageanchor=false,citecolor=blue}
\begin{titlepage}
\vspace*{7cm}
\begin{center}
{\Large Złożoność obliczeniowa }\\
\vspace*{1cm}
{\large Generated by Doxygen 1.8.1.2}\\
\vspace*{0.5cm}
{\small Sun Mar 16 2014 23:43:37}\\
\end{center}
\end{titlepage}
\clearemptydoublepage
\pagenumbering{roman}
\tableofcontents
\clearemptydoublepage
\pagenumbering{arabic}
\hypersetup{pageanchor=true,citecolor=blue}
\chapter{Program wyliczajacy czas wykonywanego alorytmu}
\label{index}\hypertarget{index}{}\begin{DoxyAuthor}{Author}
Anna Plywaczyk, 200340 
\end{DoxyAuthor}
\begin{DoxyDate}{Date}
02.\-03.\-2014 
\end{DoxyDate}
\begin{DoxyVersion}{Version}
0,1
\end{DoxyVersion}
Aplikacja porozumiewa sie z uzytkowniem. Prosi o podanie nazwy pliku, na ktorym ma byc wykonany algorytm. W programie zostanie wlaczony stoper, ktorym zmierzymy czas w jakim algorytm zostanie wykonany. Pomiar czasu zostanie wykonany kilkakrotnie i na ekran zostanie wypisana wartosc srednia. Aplikacja ponownie poprosi o podanie nazwy pliku, ktory ma zostac porownany z plikiem pomnozonym przez 2. Jezeli wektory beda jednakowe program zworoci komunikat o tym iz mnozenie zostalo wykonane poprawnie w przeciwnym wypadku zostanie zwrocony komunikat o blednym wykonaniu algorytmu. 
\chapter{Class Index}
\section{Class List}
Here are the classes, structs, unions and interfaces with brief descriptions\-:\begin{DoxyCompactList}
\item\contentsline{section}{\hyperlink{classdane}{dane} \\*Modeluje pojecie danych, u�ytych w programie, kt�re w moim przypadku sa wektorem, ktory zostaje wczytany z pliku. Pierwsza zmienna wczytana z pliku jest liczba elemntow wystepujacych w tym pliku. Przyjelam, ze zmienne wczytane z pliku sa liczbami calkowitymi (int) }{\pageref{classdane}}{}
\item\contentsline{section}{\hyperlink{classkolejkalista}{kolejkalista} \\*Klasa modeluje pojecie kolejki, bazujacej na liscie. Wykonywane sa operacje dodawnia fo kolejki, zdjecia z kolejki, sprawdzenia czy jest pusta oraz sprawdzenia rozmiaru. Tablice zaalokowalam w sposob dynamiczny }{\pageref{classkolejkalista}}{}
\item\contentsline{section}{\hyperlink{classkolejkatab}{kolejkatab} \\*Klasa modeluje pojecie kolejki, bazujacej na tablicy. Wykonywane sa operacje dodawnia do kolejki, zdjecia z kolejki, sprawdzenia czy jest pusta oraz sprawdzenia rozmiaru. Tablice zaalokowalam w sposob dynamiczny. Dodatkowo u�yte niektore funkcje pomocnicze jak np. powiekszenie tablicy }{\pageref{classkolejkatab}}{}
\item\contentsline{section}{\hyperlink{classstos}{stos} \\*Klasa modeluje pojecie stosu,bazujacego na tablicy. Na stosie wyonywane sa operacje dodatnia na stos, zdjecia ze stosu, sprawdzenia czy jest pusty oraz sprawdzenia rozmiaru. Tablice zaalokowalam w sposob dynamiczny. Dodatkowo u�yte niektore funkcje pomocnicze jak np. powiekszenie stosu }{\pageref{classstos}}{}
\item\contentsline{section}{\hyperlink{classstoslista}{stoslista} \\*Klasa modeluje pojecie stosu,bazujacego na liscie. Na stosie wyonywane sa operacje dodatnia na stos, zdjecia ze stosu, sprawdzenia czy jest pusty oraz sprawdzenia rozmiaru }{\pageref{classstoslista}}{}
\item\contentsline{section}{\hyperlink{classzegar}{zegar} \\*Klasa modeluje uruchomienia g�ownych wlasciowosci programu. Atrybutem klasy sa stowrzone dwa elemnty klasy dane, na ktorych wykonywane sa dzialania }{\pageref{classzegar}}{}
\end{DoxyCompactList}

\chapter{File Index}
\section{File List}
Here is a list of all files with brief descriptions\-:\begin{DoxyCompactList}
\item\contentsline{section}{C\-:/\-Users/\-Ania/workspace/zadanie/inc/\hyperlink{dane_8hh}{dane.\-hh} \\*Definicja klasy dane }{\pageref{dane_8hh}}{}
\item\contentsline{section}{C\-:/\-Users/\-Ania/workspace/zadanie/inc/\hyperlink{kolejka_8hh}{kolejka.\-hh} \\*Definicja klasy kolejkatab }{\pageref{kolejka_8hh}}{}
\item\contentsline{section}{C\-:/\-Users/\-Ania/workspace/zadanie/inc/\hyperlink{kolejka__lista_8hh}{kolejka\-\_\-lista.\-hh} \\*Definicja klasy kolejkalista }{\pageref{kolejka__lista_8hh}}{}
\item\contentsline{section}{C\-:/\-Users/\-Ania/workspace/zadanie/inc/\hyperlink{stos_8hh}{stos.\-hh} \\*Definicja klasy stos }{\pageref{stos_8hh}}{}
\item\contentsline{section}{C\-:/\-Users/\-Ania/workspace/zadanie/inc/\hyperlink{stos__lista_8hh}{stos\-\_\-lista.\-hh} \\*Definicja klasy stoslista }{\pageref{stos__lista_8hh}}{}
\item\contentsline{section}{C\-:/\-Users/\-Ania/workspace/zadanie/inc/\hyperlink{uruchom_8hh}{uruchom.\-hh} \\*Definicja klasy zegar }{\pageref{uruchom_8hh}}{}
\item\contentsline{section}{C\-:/\-Users/\-Ania/workspace/zadanie/src/\hyperlink{dane_8cpp}{dane.\-cpp} }{\pageref{dane_8cpp}}{}
\item\contentsline{section}{C\-:/\-Users/\-Ania/workspace/zadanie/src/\hyperlink{kolejka_8cpp}{kolejka.\-cpp} }{\pageref{kolejka_8cpp}}{}
\item\contentsline{section}{C\-:/\-Users/\-Ania/workspace/zadanie/src/\hyperlink{kolejka__lista_8cpp}{kolejka\-\_\-lista.\-cpp} }{\pageref{kolejka__lista_8cpp}}{}
\item\contentsline{section}{C\-:/\-Users/\-Ania/workspace/zadanie/src/\hyperlink{main_8cpp}{main.\-cpp} \\*Funkcja glowna ktorej glownym zalozeniem jest wczytanie plikow z rozna wielkoscia elementow znajdujacych si� w pliku, obliczenie sredniej wartosci czasu, w jakim zostaje wykonany algorytm (w naszym przypadku pomnozenie przez 2),nastepnie program porownuje poprawnosc wykoannia mnozenia z plikiem sprawdzajacym. Uzytkownik musi w programie zdefiniowac\-: liczbe powtorzen (zmienna j), ilosc plikow -\/ do ilu wykonywane jest mnozenie (zmienna i), nazwy plikow (string czesc\-\_\-1, i, czesc\-\_\-2 -\/ wszystko opcjonalnie) }{\pageref{main_8cpp}}{}
\item\contentsline{section}{C\-:/\-Users/\-Ania/workspace/zadanie/src/\hyperlink{stos_8cpp}{stos.\-cpp} }{\pageref{stos_8cpp}}{}
\item\contentsline{section}{C\-:/\-Users/\-Ania/workspace/zadanie/src/\hyperlink{stos__lista_8cpp}{stos\-\_\-lista.\-cpp} }{\pageref{stos__lista_8cpp}}{}
\item\contentsline{section}{C\-:/\-Users/\-Ania/workspace/zadanie/src/\hyperlink{uruchom_8cpp}{uruchom.\-cpp} }{\pageref{uruchom_8cpp}}{}
\end{DoxyCompactList}

\chapter{Class Documentation}
\hypertarget{classdane}{\section{dane Class Reference}
\label{classdane}\index{dane@{dane}}
}


Modeluje pojecie danych, u�ytych w programie, kt�re w moim przypadku sa wektorem, ktory zostaje wczytany z pliku. Pierwsza zmienna wczytana z pliku jest liczba elemntow wystepujacych w tym pliku. Przyjelam, ze zmienne wczytane z pliku sa liczbami calkowitymi (int).  




{\ttfamily \#include $<$dane.\-hh$>$}

\subsection*{Public Member Functions}
\begin{DoxyCompactItemize}
\item 
void \hyperlink{classdane_a9d11db4780438d62db6e12060c2c84f0}{wczytaj} (string nazwa)
\begin{DoxyCompactList}\small\item\em Funkcja wczytuj�ca dane do wektora z pliku. Funkcja otwiera plik zdefiniowany w glownej funkcji przez u�ytkownika, sprawdza czy plik zosta� otwarty, je�eli zosta� otwarty wczytana jest liczba elementow pliku, nastepnie wczytywane sa wszystkie liczby do wektora. \end{DoxyCompactList}\item 
void \hyperlink{classdane_a8c2265d257e8915ab0927da72d1eda59}{wypisz} ()
\begin{DoxyCompactList}\small\item\em Funkcja wypisuj�ca na ekran wszystkie elementy wektora. Funkcja wypisuje na ekran wsyztskie elemnty z pliku na ekran w postaci wektora w kolumnie. \end{DoxyCompactList}\item 
void \hyperlink{classdane_a3e9a764d9affa78d3f6b01d77996d3ec}{zamien\-\_\-element} (int i, int j)
\begin{DoxyCompactList}\small\item\em Zamiana elementow Funkcja zamiania dwa elementy wektora, zadane poprzez wywolanie argumentow funkcji. \end{DoxyCompactList}\item 
void \hyperlink{classdane_a1bd757a9b99595578ca23f52ea036f20}{dodaj\-\_\-element} (int e)
\begin{DoxyCompactList}\small\item\em Dodawanie elementu. Funkcja dodaje element na koniec wektora. Funkcja zdefiniowana jest poprzez wywolanie argumentow funkcji. \end{DoxyCompactList}\item 
void \hyperlink{classdane_a707fcfe2661ea4c59e4c9be456fc96d7}{odwroc\-\_\-element} ()
\begin{DoxyCompactList}\small\item\em Odwracanie elementow Funkcja ogawraca wektor, ostatni element wektora staje si� pierwszy, a pierwszy ostatnim. \end{DoxyCompactList}\item 
void \hyperlink{classdane_a257a2d91e163464fe7fb8c1a7adb1728}{dodaj\-\_\-elementy} (\hyperlink{classdane}{dane} wektor2)
\begin{DoxyCompactList}\small\item\em Dodawanie elementu. Funkcja dodaje element na koniec wektora. Funkcja zdefiniowana jest poprzez wywolanie argumentow funkcji. \end{DoxyCompactList}\item 
int \& \hyperlink{classdane_af5f6aad7a57c9709cfdd50bc57c9ecb3}{operator\mbox{[}$\,$\mbox{]}} (int indeks)
\begin{DoxyCompactList}\small\item\em Uzycie operatora \mbox{[}\mbox{]} Przeciazenie operatora stwoarzone abysmy mogli odwolac sie do konkretnego elementu wektora. \end{DoxyCompactList}\item 
unsigned int \hyperlink{classdane_aad79f522f96d20e05e7e84db96dfe162}{size} ()
\begin{DoxyCompactList}\small\item\em Rozmiar wektora. \end{DoxyCompactList}\item 
\hyperlink{classdane}{dane} \& \hyperlink{classdane_a0316457927e41afa079b84ed878c7462}{operator+} (\hyperlink{classdane}{dane} wektor2)
\begin{DoxyCompactList}\small\item\em Uzycie operatora + na wektorze Przeciazenie operatora dodwania, ktory mozemy wykonywac na wektorach. \end{DoxyCompactList}\item 
\hyperlink{classdane}{dane} \& \hyperlink{classdane_a2b3162017b009396e52273f80b1e43a2}{operator=} (\hyperlink{classdane}{dane} wektor2)
\begin{DoxyCompactList}\small\item\em Uzycie operatora = na wektorze Przeciazenie opertora przypisywania. \end{DoxyCompactList}\item 
vector$<$ int $>$ \& \hyperlink{classdane_adc96a79286edf951110bcb1206214912}{wnetrze} ()
\begin{DoxyCompactList}\small\item\em Metoda daj�ca dostep do zawartosci wektora danych. \end{DoxyCompactList}\item 
bool \hyperlink{classdane_a0024956f16e82184d7da0650d4a3372a}{operator==} (\hyperlink{classdane}{dane} wektor2)
\begin{DoxyCompactList}\small\item\em Operator porownania dwoch wektorow Funkcja, ktora jest operatorem porownania dwoch wektorow. \end{DoxyCompactList}\end{DoxyCompactItemize}


\subsection{Detailed Description}
Modeluje pojecie danych, u�ytych w programie, kt�re w moim przypadku sa wektorem, ktory zostaje wczytany z pliku. Pierwsza zmienna wczytana z pliku jest liczba elemntow wystepujacych w tym pliku. Przyjelam, ze zmienne wczytane z pliku sa liczbami calkowitymi (int). 

Definition at line 31 of file dane.\-hh.



\subsection{Member Function Documentation}
\hypertarget{classdane_a1bd757a9b99595578ca23f52ea036f20}{\index{dane@{dane}!dodaj\-\_\-element@{dodaj\-\_\-element}}
\index{dodaj\-\_\-element@{dodaj\-\_\-element}!dane@{dane}}
\subsubsection[{dodaj\-\_\-element}]{\setlength{\rightskip}{0pt plus 5cm}void dane\-::dodaj\-\_\-element (
\begin{DoxyParamCaption}
\item[{int}]{e}
\end{DoxyParamCaption}
)}}\label{classdane_a1bd757a9b99595578ca23f52ea036f20}


Dodawanie elementu. Funkcja dodaje element na koniec wektora. Funkcja zdefiniowana jest poprzez wywolanie argumentow funkcji. 


\begin{DoxyParams}[1]{Parameters}
\mbox{\tt in}  & {\em e} & -\/ liczba, ktora ma zostac dodana na koniec wektora. return (brak) \\
\hline
\end{DoxyParams}


Definition at line 49 of file dane.\-cpp.

\hypertarget{classdane_a257a2d91e163464fe7fb8c1a7adb1728}{\index{dane@{dane}!dodaj\-\_\-elementy@{dodaj\-\_\-elementy}}
\index{dodaj\-\_\-elementy@{dodaj\-\_\-elementy}!dane@{dane}}
\subsubsection[{dodaj\-\_\-elementy}]{\setlength{\rightskip}{0pt plus 5cm}void dane\-::dodaj\-\_\-elementy (
\begin{DoxyParamCaption}
\item[{{\bf dane}}]{wektor2}
\end{DoxyParamCaption}
)}}\label{classdane_a257a2d91e163464fe7fb8c1a7adb1728}


Dodawanie elementu. Funkcja dodaje element na koniec wektora. Funkcja zdefiniowana jest poprzez wywolanie argumentow funkcji. 


\begin{DoxyParams}[1]{Parameters}
\mbox{\tt in}  & {\em wektor2} & -\/ wektor, ktory ma zostac dodany na koniec wektora, z dwoch zostaje stworzony jeden. return (brak) \\
\hline
\end{DoxyParams}


Definition at line 66 of file dane.\-cpp.

\hypertarget{classdane_a707fcfe2661ea4c59e4c9be456fc96d7}{\index{dane@{dane}!odwroc\-\_\-element@{odwroc\-\_\-element}}
\index{odwroc\-\_\-element@{odwroc\-\_\-element}!dane@{dane}}
\subsubsection[{odwroc\-\_\-element}]{\setlength{\rightskip}{0pt plus 5cm}void dane\-::odwroc\-\_\-element (
\begin{DoxyParamCaption}
{}
\end{DoxyParamCaption}
)}}\label{classdane_a707fcfe2661ea4c59e4c9be456fc96d7}


Odwracanie elementow Funkcja ogawraca wektor, ostatni element wektora staje si� pierwszy, a pierwszy ostatnim. 

\begin{DoxyReturn}{Returns}
(brak) 
\end{DoxyReturn}


Definition at line 55 of file dane.\-cpp.

\hypertarget{classdane_a0316457927e41afa079b84ed878c7462}{\index{dane@{dane}!operator+@{operator+}}
\index{operator+@{operator+}!dane@{dane}}
\subsubsection[{operator+}]{\setlength{\rightskip}{0pt plus 5cm}{\bf dane} \& dane\-::operator+ (
\begin{DoxyParamCaption}
\item[{{\bf dane}}]{wektor2}
\end{DoxyParamCaption}
)}}\label{classdane_a0316457927e41afa079b84ed878c7462}


Uzycie operatora + na wektorze Przeciazenie operatora dodwania, ktory mozemy wykonywac na wektorach. 


\begin{DoxyParams}[1]{Parameters}
\mbox{\tt in}  & {\em wektor2} & -\/ wektor danych ktory ma zostac dodany do wektora glownego \\
\hline
\end{DoxyParams}
\begin{DoxyReturn}{Returns}
Zwraca wektor, ktory jest suma dwoch innych 
\end{DoxyReturn}


Definition at line 79 of file dane.\-cpp.

\hypertarget{classdane_a2b3162017b009396e52273f80b1e43a2}{\index{dane@{dane}!operator=@{operator=}}
\index{operator=@{operator=}!dane@{dane}}
\subsubsection[{operator=}]{\setlength{\rightskip}{0pt plus 5cm}{\bf dane} \& dane\-::operator= (
\begin{DoxyParamCaption}
\item[{{\bf dane}}]{wektor2}
\end{DoxyParamCaption}
)}}\label{classdane_a2b3162017b009396e52273f80b1e43a2}


Uzycie operatora = na wektorze Przeciazenie opertora przypisywania. 


\begin{DoxyParams}[1]{Parameters}
\mbox{\tt in}  & {\em wektor2} & -\/ wektor danych ktory ma zostac przypisany do wektora glownego \\
\hline
\end{DoxyParams}
\begin{DoxyReturn}{Returns}
Zwraca wektor, do ktorego zostal przypisany inny wektor 
\end{DoxyReturn}


Definition at line 85 of file dane.\-cpp.

\hypertarget{classdane_a0024956f16e82184d7da0650d4a3372a}{\index{dane@{dane}!operator==@{operator==}}
\index{operator==@{operator==}!dane@{dane}}
\subsubsection[{operator==}]{\setlength{\rightskip}{0pt plus 5cm}bool dane\-::operator== (
\begin{DoxyParamCaption}
\item[{{\bf dane}}]{wektor2}
\end{DoxyParamCaption}
)}}\label{classdane_a0024956f16e82184d7da0650d4a3372a}


Operator porownania dwoch wektorow Funkcja, ktora jest operatorem porownania dwoch wektorow. 


\begin{DoxyParams}[1]{Parameters}
\mbox{\tt in}  & {\em wektor2} & -\/ wektor danych ktory zostaje porownany z danymi glownymi \\
\hline
\end{DoxyParams}
\begin{DoxyReturn}{Returns}
True gdy wektory danych sa jednakowe, natomiast jesli nawet jeden element sie rozni od wektora porownywanego zwraca false. 
\end{DoxyReturn}


Definition at line 93 of file dane.\-cpp.

\hypertarget{classdane_af5f6aad7a57c9709cfdd50bc57c9ecb3}{\index{dane@{dane}!operator\mbox{[}$\,$\mbox{]}@{operator[]}}
\index{operator\mbox{[}$\,$\mbox{]}@{operator[]}!dane@{dane}}
\subsubsection[{operator[]}]{\setlength{\rightskip}{0pt plus 5cm}int \& dane\-::operator\mbox{[}$\,$\mbox{]} (
\begin{DoxyParamCaption}
\item[{int}]{indeks}
\end{DoxyParamCaption}
)}}\label{classdane_af5f6aad7a57c9709cfdd50bc57c9ecb3}


Uzycie operatora \mbox{[}\mbox{]} Przeciazenie operatora stwoarzone abysmy mogli odwolac sie do konkretnego elementu wektora. 


\begin{DoxyParams}[1]{Parameters}
\mbox{\tt in}  & {\em indeks} & -\/ zmienna calkowita, poperzez ktora mozemy dostac do konkretnego elementu wektora. \\
\hline
\end{DoxyParams}
\begin{DoxyReturn}{Returns}
Zwraca wartosc jaka znajduje si� w zadanym elemencie wektora. 
\end{DoxyReturn}


Definition at line 74 of file dane.\-cpp.

\hypertarget{classdane_aad79f522f96d20e05e7e84db96dfe162}{\index{dane@{dane}!size@{size}}
\index{size@{size}!dane@{dane}}
\subsubsection[{size}]{\setlength{\rightskip}{0pt plus 5cm}unsigned int dane\-::size (
\begin{DoxyParamCaption}
{}
\end{DoxyParamCaption}
)\hspace{0.3cm}{\ttfamily [inline]}}}\label{classdane_aad79f522f96d20e05e7e84db96dfe162}


Rozmiar wektora. 

\begin{DoxyReturn}{Returns}
Funkcja zwaraca liczbe elementow wektora. 
\end{DoxyReturn}


Definition at line 102 of file dane.\-hh.

\hypertarget{classdane_a9d11db4780438d62db6e12060c2c84f0}{\index{dane@{dane}!wczytaj@{wczytaj}}
\index{wczytaj@{wczytaj}!dane@{dane}}
\subsubsection[{wczytaj}]{\setlength{\rightskip}{0pt plus 5cm}void dane\-::wczytaj (
\begin{DoxyParamCaption}
\item[{string}]{nazwa}
\end{DoxyParamCaption}
)}}\label{classdane_a9d11db4780438d62db6e12060c2c84f0}


Funkcja wczytuj�ca dane do wektora z pliku. Funkcja otwiera plik zdefiniowany w glownej funkcji przez u�ytkownika, sprawdza czy plik zosta� otwarty, je�eli zosta� otwarty wczytana jest liczba elementow pliku, nastepnie wczytywane sa wszystkie liczby do wektora. 


\begin{DoxyParams}[1]{Parameters}
\mbox{\tt in}  & {\em nazwa} & -\/ zmienna, kt�ra zostaje wprowadzona przez u�ytkownika do programu \\
\hline
\end{DoxyParams}
\begin{DoxyReturn}{Returns}
(brak) 
\end{DoxyReturn}


Definition at line 12 of file dane.\-cpp.

\hypertarget{classdane_adc96a79286edf951110bcb1206214912}{\index{dane@{dane}!wnetrze@{wnetrze}}
\index{wnetrze@{wnetrze}!dane@{dane}}
\subsubsection[{wnetrze}]{\setlength{\rightskip}{0pt plus 5cm}vector$<$int$>$\& dane\-::wnetrze (
\begin{DoxyParamCaption}
{}
\end{DoxyParamCaption}
)\hspace{0.3cm}{\ttfamily [inline]}}}\label{classdane_adc96a79286edf951110bcb1206214912}


Metoda daj�ca dostep do zawartosci wektora danych. 

\begin{DoxyReturn}{Returns}
Wektor danych. 
\end{DoxyReturn}


Definition at line 123 of file dane.\-hh.

\hypertarget{classdane_a8c2265d257e8915ab0927da72d1eda59}{\index{dane@{dane}!wypisz@{wypisz}}
\index{wypisz@{wypisz}!dane@{dane}}
\subsubsection[{wypisz}]{\setlength{\rightskip}{0pt plus 5cm}void dane\-::wypisz (
\begin{DoxyParamCaption}
{}
\end{DoxyParamCaption}
)}}\label{classdane_a8c2265d257e8915ab0927da72d1eda59}


Funkcja wypisuj�ca na ekran wszystkie elementy wektora. Funkcja wypisuje na ekran wsyztskie elemnty z pliku na ekran w postaci wektora w kolumnie. 

\begin{DoxyReturn}{Returns}
(brak) 
\end{DoxyReturn}


Definition at line 29 of file dane.\-cpp.

\hypertarget{classdane_a3e9a764d9affa78d3f6b01d77996d3ec}{\index{dane@{dane}!zamien\-\_\-element@{zamien\-\_\-element}}
\index{zamien\-\_\-element@{zamien\-\_\-element}!dane@{dane}}
\subsubsection[{zamien\-\_\-element}]{\setlength{\rightskip}{0pt plus 5cm}void dane\-::zamien\-\_\-element (
\begin{DoxyParamCaption}
\item[{int}]{i, }
\item[{int}]{j}
\end{DoxyParamCaption}
)}}\label{classdane_a3e9a764d9affa78d3f6b01d77996d3ec}


Zamiana elementow Funkcja zamiania dwa elementy wektora, zadane poprzez wywolanie argumentow funkcji. 


\begin{DoxyParams}[1]{Parameters}
\mbox{\tt in}  & {\em i} & -\/ pierwszy numer elementu, ktory ma zostac zamieniony. \\
\hline
\mbox{\tt in}  & {\em j-\/} & drugi numer elementu, ktory ma zostac zamieniony. \\
\hline
\end{DoxyParams}
\begin{DoxyReturn}{Returns}
(brak) 
\end{DoxyReturn}


Definition at line 37 of file dane.\-cpp.



The documentation for this class was generated from the following files\-:\begin{DoxyCompactItemize}
\item 
C\-:/\-Users/\-Ania/workspace/zadanie/inc/\hyperlink{dane_8hh}{dane.\-hh}\item 
C\-:/\-Users/\-Ania/workspace/zadanie/src/\hyperlink{dane_8cpp}{dane.\-cpp}\end{DoxyCompactItemize}

\hypertarget{classkolejkalista}{\section{kolejkalista Class Reference}
\label{classkolejkalista}\index{kolejkalista@{kolejkalista}}
}


Klasa modeluje pojecie kolejki, bazujacej na liscie. Wykonywane sa operacje dodawnia fo kolejki, zdjecia z kolejki, sprawdzenia czy jest pusta oraz sprawdzenia rozmiaru. Tablice zaalokowalam w sposob dynamiczny.  




{\ttfamily \#include $<$kolejka\-\_\-lista.\-hh$>$}

\subsection*{Public Member Functions}
\begin{DoxyCompactItemize}
\item 
void \hyperlink{classkolejkalista_a32d0ea8aed298035a84f2bd1fedea0ef}{enqueue} (int el\-\_\-dodawany)
\begin{DoxyCompactList}\small\item\em Dodanie elementu z kolejki Funkcja dodaje element na poczatek kolejki. \end{DoxyCompactList}\item 
void \hyperlink{classkolejkalista_a31d4f038aafc4472ab2eebf20021820a}{dequeue} (int $\ast$a)
\begin{DoxyCompactList}\small\item\em Zdejmowanie elementu z kolejki. Funkcja zdejmuje element z konca kolejki. \end{DoxyCompactList}\item 
bool \hyperlink{classkolejkalista_ac3e0509cda7ba208478ecf3c5bbabd14}{isempty} ()
\begin{DoxyCompactList}\small\item\em Sprawdzanie pojemnosci w kolejce. Funkcja sprawdza czy jest pusta poprzez porownanie liczby\-\_\-elementow do 0. \end{DoxyCompactList}\item 
int \hyperlink{classkolejkalista_a2f53de089b8a97474de6b42b1da04928}{size} ()
\begin{DoxyCompactList}\small\item\em Zrowcenie rozmiaru funkcji. Funkcja sprawdza ile elementow jest w kolejce. \end{DoxyCompactList}\end{DoxyCompactItemize}


\subsection{Detailed Description}
Klasa modeluje pojecie kolejki, bazujacej na liscie. Wykonywane sa operacje dodawnia fo kolejki, zdjecia z kolejki, sprawdzenia czy jest pusta oraz sprawdzenia rozmiaru. Tablice zaalokowalam w sposob dynamiczny. 

Definition at line 29 of file kolejka\-\_\-lista.\-hh.



\subsection{Member Function Documentation}
\hypertarget{classkolejkalista_a31d4f038aafc4472ab2eebf20021820a}{\index{kolejkalista@{kolejkalista}!dequeue@{dequeue}}
\index{dequeue@{dequeue}!kolejkalista@{kolejkalista}}
\subsubsection[{dequeue}]{\setlength{\rightskip}{0pt plus 5cm}void kolejkalista\-::dequeue (
\begin{DoxyParamCaption}
\item[{int $\ast$}]{a}
\end{DoxyParamCaption}
)}}\label{classkolejkalista_a31d4f038aafc4472ab2eebf20021820a}


Zdejmowanie elementu z kolejki. Funkcja zdejmuje element z konca kolejki. 


\begin{DoxyParams}[1]{Parameters}
\mbox{\tt in}  & {\em a} & -\/ wskaznik do ktorego przypisywany jest element zdejmowany, aby mogl zostal uzyty w przyszlosci \\
\hline
\end{DoxyParams}
\begin{DoxyReturn}{Returns}
(brak) 
\end{DoxyReturn}


Definition at line 18 of file kolejka\-\_\-lista.\-cpp.

\hypertarget{classkolejkalista_a32d0ea8aed298035a84f2bd1fedea0ef}{\index{kolejkalista@{kolejkalista}!enqueue@{enqueue}}
\index{enqueue@{enqueue}!kolejkalista@{kolejkalista}}
\subsubsection[{enqueue}]{\setlength{\rightskip}{0pt plus 5cm}void kolejkalista\-::enqueue (
\begin{DoxyParamCaption}
\item[{int}]{el\-\_\-dodawany}
\end{DoxyParamCaption}
)}}\label{classkolejkalista_a32d0ea8aed298035a84f2bd1fedea0ef}


Dodanie elementu z kolejki Funkcja dodaje element na poczatek kolejki. 


\begin{DoxyParams}[1]{Parameters}
\mbox{\tt in}  & {\em el\-\_\-dodawany} & -\/ zmienna stala, ktora ma zostac dodana poczatek kolejki \\
\hline
\end{DoxyParams}
\begin{DoxyReturn}{Returns}
(brak) 
\end{DoxyReturn}


Definition at line 14 of file kolejka\-\_\-lista.\-cpp.

\hypertarget{classkolejkalista_ac3e0509cda7ba208478ecf3c5bbabd14}{\index{kolejkalista@{kolejkalista}!isempty@{isempty}}
\index{isempty@{isempty}!kolejkalista@{kolejkalista}}
\subsubsection[{isempty}]{\setlength{\rightskip}{0pt plus 5cm}bool kolejkalista\-::isempty (
\begin{DoxyParamCaption}
{}
\end{DoxyParamCaption}
)}}\label{classkolejkalista_ac3e0509cda7ba208478ecf3c5bbabd14}


Sprawdzanie pojemnosci w kolejce. Funkcja sprawdza czy jest pusta poprzez porownanie liczby\-\_\-elementow do 0. 

\begin{DoxyReturn}{Returns}
true jezeli funkcja jest pusta, w przeciwnym wypadku false. 
\end{DoxyReturn}


Definition at line 30 of file kolejka\-\_\-lista.\-cpp.

\hypertarget{classkolejkalista_a2f53de089b8a97474de6b42b1da04928}{\index{kolejkalista@{kolejkalista}!size@{size}}
\index{size@{size}!kolejkalista@{kolejkalista}}
\subsubsection[{size}]{\setlength{\rightskip}{0pt plus 5cm}int kolejkalista\-::size (
\begin{DoxyParamCaption}
{}
\end{DoxyParamCaption}
)}}\label{classkolejkalista_a2f53de089b8a97474de6b42b1da04928}


Zrowcenie rozmiaru funkcji. Funkcja sprawdza ile elementow jest w kolejce. 

\begin{DoxyReturn}{Returns}
Zwraca liczbe elementow. 
\end{DoxyReturn}


Definition at line 34 of file kolejka\-\_\-lista.\-cpp.



The documentation for this class was generated from the following files\-:\begin{DoxyCompactItemize}
\item 
C\-:/\-Users/\-Ania/workspace/zadanie/inc/\hyperlink{kolejka__lista_8hh}{kolejka\-\_\-lista.\-hh}\item 
C\-:/\-Users/\-Ania/workspace/zadanie/src/\hyperlink{kolejka__lista_8cpp}{kolejka\-\_\-lista.\-cpp}\end{DoxyCompactItemize}

\hypertarget{classkolejkatab}{\section{kolejkatab Class Reference}
\label{classkolejkatab}\index{kolejkatab@{kolejkatab}}
}


Klasa modeluje pojecie kolejki, bazujacej na tablicy. Wykonywane sa operacje dodawnia do kolejki, zdjecia z kolejki, sprawdzenia czy jest pusta oraz sprawdzenia rozmiaru. Tablice zaalokowalam w sposob dynamiczny. Dodatkowo u�yte niektore funkcje pomocnicze jak np. powiekszenie tablicy.  




{\ttfamily \#include $<$kolejka.\-hh$>$}

\subsection*{Public Member Functions}
\begin{DoxyCompactItemize}
\item 
\hyperlink{classkolejkatab_a1f7e611a459f7717d78f32f18091cd70}{kolejkatab} ()
\begin{DoxyCompactList}\small\item\em Konstruktor bezargumentowy. Przy tworzeniu tablica jest pusta, w konstruktorze przydzielana jest dynamicznie, na poczatku tablica ma mozliwosc przypisania tablicy o rozmiarze 1. Jezeli nie uda sie utworzyc tablicy wyrzucany jest blad. Przy wywo�aniu konstrktora bezparametrycznego powiekszajaca tablica bedzie dwa razy wieksza. \end{DoxyCompactList}\item 
void \hyperlink{classkolejkatab_a2bd13cb070ac78c1fe546c670227c8df}{enqueue} (int el\-\_\-dodawany)
\begin{DoxyCompactList}\small\item\em Dodanie elementu z kolejki Funkcja dodaje element na poczatek kolejki, jezeli brakuje miejsca podwaja pamiec. \end{DoxyCompactList}\item 
void \hyperlink{classkolejkatab_a90b01b396cbd0d943240206a7eab1fea}{dequeue} (int $\ast$a)
\begin{DoxyCompactList}\small\item\em Zdejmowanie elementu z kolejki. Funkcja zdejmuje element z konca kolejki, jezeli liczba elementow bedzie mniejsza badz rowna 1/4 mozliwych elementow w tablicy, zostaje ona pomniejszana. \end{DoxyCompactList}\item 
bool \hyperlink{classkolejkatab_a8c132b787bfaf99734a03bae767d756c}{isempty} ()
\begin{DoxyCompactList}\small\item\em Sprawdzanie pojemnosci w kolejce. Funkcja sprawdza czy jest pusta poprzez porownanie liczby\-\_\-elementow do 0. \end{DoxyCompactList}\item 
int \hyperlink{classkolejkatab_a98be83a96f21b460b9baf73d92619063}{size} ()
\begin{DoxyCompactList}\small\item\em Sprawdzenie rozmiaru funkcji Funkcja sprawdza ile elementow jest w kolejce. \end{DoxyCompactList}\end{DoxyCompactItemize}


\subsection{Detailed Description}
Klasa modeluje pojecie kolejki, bazujacej na tablicy. Wykonywane sa operacje dodawnia do kolejki, zdjecia z kolejki, sprawdzenia czy jest pusta oraz sprawdzenia rozmiaru. Tablice zaalokowalam w sposob dynamiczny. Dodatkowo u�yte niektore funkcje pomocnicze jak np. powiekszenie tablicy. 

Definition at line 29 of file kolejka.\-hh.



\subsection{Constructor \& Destructor Documentation}
\hypertarget{classkolejkatab_a1f7e611a459f7717d78f32f18091cd70}{\index{kolejkatab@{kolejkatab}!kolejkatab@{kolejkatab}}
\index{kolejkatab@{kolejkatab}!kolejkatab@{kolejkatab}}
\subsubsection[{kolejkatab}]{\setlength{\rightskip}{0pt plus 5cm}kolejkatab\-::kolejkatab (
\begin{DoxyParamCaption}
{}
\end{DoxyParamCaption}
)\hspace{0.3cm}{\ttfamily [inline]}}}\label{classkolejkatab_a1f7e611a459f7717d78f32f18091cd70}


Konstruktor bezargumentowy. Przy tworzeniu tablica jest pusta, w konstruktorze przydzielana jest dynamicznie, na poczatku tablica ma mozliwosc przypisania tablicy o rozmiarze 1. Jezeli nie uda sie utworzyc tablicy wyrzucany jest blad. Przy wywo�aniu konstrktora bezparametrycznego powiekszajaca tablica bedzie dwa razy wieksza. 



Definition at line 79 of file kolejka.\-hh.



\subsection{Member Function Documentation}
\hypertarget{classkolejkatab_a90b01b396cbd0d943240206a7eab1fea}{\index{kolejkatab@{kolejkatab}!dequeue@{dequeue}}
\index{dequeue@{dequeue}!kolejkatab@{kolejkatab}}
\subsubsection[{dequeue}]{\setlength{\rightskip}{0pt plus 5cm}void kolejkatab\-::dequeue (
\begin{DoxyParamCaption}
\item[{int $\ast$}]{a}
\end{DoxyParamCaption}
)}}\label{classkolejkatab_a90b01b396cbd0d943240206a7eab1fea}


Zdejmowanie elementu z kolejki. Funkcja zdejmuje element z konca kolejki, jezeli liczba elementow bedzie mniejsza badz rowna 1/4 mozliwych elementow w tablicy, zostaje ona pomniejszana. 


\begin{DoxyParams}[1]{Parameters}
\mbox{\tt in}  & {\em a} & -\/ wskaznik do ktorego przypisywany jest element zdejmowany, aby mogl zostal uzyty w przyszlosci \\
\hline
\end{DoxyParams}
\begin{DoxyReturn}{Returns}
(brak) 
\end{DoxyReturn}


Definition at line 58 of file kolejka.\-cpp.

\hypertarget{classkolejkatab_a2bd13cb070ac78c1fe546c670227c8df}{\index{kolejkatab@{kolejkatab}!enqueue@{enqueue}}
\index{enqueue@{enqueue}!kolejkatab@{kolejkatab}}
\subsubsection[{enqueue}]{\setlength{\rightskip}{0pt plus 5cm}void kolejkatab\-::enqueue (
\begin{DoxyParamCaption}
\item[{int}]{el\-\_\-dodawany}
\end{DoxyParamCaption}
)}}\label{classkolejkatab_a2bd13cb070ac78c1fe546c670227c8df}


Dodanie elementu z kolejki Funkcja dodaje element na poczatek kolejki, jezeli brakuje miejsca podwaja pamiec. 


\begin{DoxyParams}[1]{Parameters}
\mbox{\tt in}  & {\em el\-\_\-dodawany} & -\/ zmienna stala, ktora ma zostac dodana na koniec stosu \\
\hline
\end{DoxyParams}
\begin{DoxyReturn}{Returns}
(brak) 
\end{DoxyReturn}


Definition at line 39 of file kolejka.\-cpp.

\hypertarget{classkolejkatab_a8c132b787bfaf99734a03bae767d756c}{\index{kolejkatab@{kolejkatab}!isempty@{isempty}}
\index{isempty@{isempty}!kolejkatab@{kolejkatab}}
\subsubsection[{isempty}]{\setlength{\rightskip}{0pt plus 5cm}bool kolejkatab\-::isempty (
\begin{DoxyParamCaption}
{}
\end{DoxyParamCaption}
)}}\label{classkolejkatab_a8c132b787bfaf99734a03bae767d756c}


Sprawdzanie pojemnosci w kolejce. Funkcja sprawdza czy jest pusta poprzez porownanie liczby\-\_\-elementow do 0. 

\begin{DoxyReturn}{Returns}
true jezeli funkcja jest pusta, w przeciwnym wypadku false. 
\end{DoxyReturn}


Definition at line 73 of file kolejka.\-cpp.

\hypertarget{classkolejkatab_a98be83a96f21b460b9baf73d92619063}{\index{kolejkatab@{kolejkatab}!size@{size}}
\index{size@{size}!kolejkatab@{kolejkatab}}
\subsubsection[{size}]{\setlength{\rightskip}{0pt plus 5cm}int kolejkatab\-::size (
\begin{DoxyParamCaption}
{}
\end{DoxyParamCaption}
)}}\label{classkolejkatab_a98be83a96f21b460b9baf73d92619063}


Sprawdzenie rozmiaru funkcji Funkcja sprawdza ile elementow jest w kolejce. 

\begin{DoxyReturn}{Returns}
(brak) 
\end{DoxyReturn}


Definition at line 77 of file kolejka.\-cpp.



The documentation for this class was generated from the following files\-:\begin{DoxyCompactItemize}
\item 
C\-:/\-Users/\-Ania/workspace/zadanie/inc/\hyperlink{kolejka_8hh}{kolejka.\-hh}\item 
C\-:/\-Users/\-Ania/workspace/zadanie/src/\hyperlink{kolejka_8cpp}{kolejka.\-cpp}\end{DoxyCompactItemize}

\hypertarget{classstos}{\section{stos Class Reference}
\label{classstos}\index{stos@{stos}}
}


Klasa modeluje pojecie stosu,bazujacego na tablicy. Na stosie wyonywane sa operacje dodatnia na stos, zdjecia ze stosu, sprawdzenia czy jest pusty oraz sprawdzenia rozmiaru. Tablice zaalokowalam w sposob dynamiczny. Dodatkowo u�yte niektore funkcje pomocnicze jak np. powiekszenie stosu.  




{\ttfamily \#include $<$stos.\-hh$>$}

\subsection*{Public Member Functions}
\begin{DoxyCompactItemize}
\item 
\hyperlink{classstos_afb387ac69250038334f6d8099b6a2421}{stos} ()
\begin{DoxyCompactList}\small\item\em Konstruktor bezargumentowy. Przy tworzeniu tablica jest pusta, w konstruktorze przydzielana jest dynamicznie, na poczatku tablica ma mozliwosc przypisania tablicy o rozmiarze 1. Jezeli nie uda sie utworzyc tablicy wyrzucany jest blad. Przy wywo�aniu konstrktora bezparametrycznego powiekszajaca tablica bedzie dwa razy wieksza. \end{DoxyCompactList}\item 
\hyperlink{classstos_acd9aed33787d883581056faa8442b8b8}{stos} (int cos)
\begin{DoxyCompactList}\small\item\em Konstruktor parametryczny. Przy tworzeniu tablica jest pusta, w konstruktorze przydzielana jest dynamicznie, na poczatku tablica ma mozliwosc przypisania tablicy o rozmiarze 1. Jezeli nie uda sie utworzyc tablicy wyrzucany jest blad. Przy wywo�aniu konstrktora parametrycznego tablica zostanie powiekszona o 1 element. \end{DoxyCompactList}\item 
void \hyperlink{classstos_ae3a9afe12a83444555ce0246233fa9e0}{push} (int el\-\_\-dodawany)
\begin{DoxyCompactList}\small\item\em Dodanie elementu na stos. Funkcja dodaje element na koniec stosu, jezeli brakuje miejsca powieksza pamiec. \end{DoxyCompactList}\item 
void \hyperlink{classstos_a2ee24ebe7ef28d8fb0e37950931a548f}{pop} (int $\ast$a)
\begin{DoxyCompactList}\small\item\em Zdejmowanie elementu ze stosu. Funkcja zdejmuje element z konca stosu, jezeli liczba elementow bedzie mniejsza badz rowna 1/4 mozliwych elementow w tablicy, zostaje ona pomniejszana. \end{DoxyCompactList}\item 
bool \hyperlink{classstos_a28f9b17c72b4a91221e4bae54e169895}{isempty} ()
\begin{DoxyCompactList}\small\item\em Zwrocenie rozmiaru stosu Funkcja sprawdza czy stos jest pusty poprzez porownanie liczby\-\_\-elementow do 0. \end{DoxyCompactList}\item 
int \hyperlink{classstos_a47fcfe525e580ceb48080b33ab3d53de}{size} ()
\begin{DoxyCompactList}\small\item\em Sprawdzenie rozmiaru funkcji Funkcja sprawdza ile elementow jest na stosie. \end{DoxyCompactList}\end{DoxyCompactItemize}


\subsection{Detailed Description}
Klasa modeluje pojecie stosu,bazujacego na tablicy. Na stosie wyonywane sa operacje dodatnia na stos, zdjecia ze stosu, sprawdzenia czy jest pusty oraz sprawdzenia rozmiaru. Tablice zaalokowalam w sposob dynamiczny. Dodatkowo u�yte niektore funkcje pomocnicze jak np. powiekszenie stosu. 

Definition at line 29 of file stos.\-hh.



\subsection{Constructor \& Destructor Documentation}
\hypertarget{classstos_afb387ac69250038334f6d8099b6a2421}{\index{stos@{stos}!stos@{stos}}
\index{stos@{stos}!stos@{stos}}
\subsubsection[{stos}]{\setlength{\rightskip}{0pt plus 5cm}stos\-::stos (
\begin{DoxyParamCaption}
{}
\end{DoxyParamCaption}
)\hspace{0.3cm}{\ttfamily [inline]}}}\label{classstos_afb387ac69250038334f6d8099b6a2421}


Konstruktor bezargumentowy. Przy tworzeniu tablica jest pusta, w konstruktorze przydzielana jest dynamicznie, na poczatku tablica ma mozliwosc przypisania tablicy o rozmiarze 1. Jezeli nie uda sie utworzyc tablicy wyrzucany jest blad. Przy wywo�aniu konstrktora bezparametrycznego powiekszajaca tablica bedzie dwa razy wieksza. 



Definition at line 79 of file stos.\-hh.

\hypertarget{classstos_acd9aed33787d883581056faa8442b8b8}{\index{stos@{stos}!stos@{stos}}
\index{stos@{stos}!stos@{stos}}
\subsubsection[{stos}]{\setlength{\rightskip}{0pt plus 5cm}stos\-::stos (
\begin{DoxyParamCaption}
\item[{int}]{cos}
\end{DoxyParamCaption}
)\hspace{0.3cm}{\ttfamily [inline]}}}\label{classstos_acd9aed33787d883581056faa8442b8b8}


Konstruktor parametryczny. Przy tworzeniu tablica jest pusta, w konstruktorze przydzielana jest dynamicznie, na poczatku tablica ma mozliwosc przypisania tablicy o rozmiarze 1. Jezeli nie uda sie utworzyc tablicy wyrzucany jest blad. Przy wywo�aniu konstrktora parametrycznego tablica zostanie powiekszona o 1 element. 



Definition at line 96 of file stos.\-hh.



\subsection{Member Function Documentation}
\hypertarget{classstos_a28f9b17c72b4a91221e4bae54e169895}{\index{stos@{stos}!isempty@{isempty}}
\index{isempty@{isempty}!stos@{stos}}
\subsubsection[{isempty}]{\setlength{\rightskip}{0pt plus 5cm}bool stos\-::isempty (
\begin{DoxyParamCaption}
{}
\end{DoxyParamCaption}
)}}\label{classstos_a28f9b17c72b4a91221e4bae54e169895}


Zwrocenie rozmiaru stosu Funkcja sprawdza czy stos jest pusty poprzez porownanie liczby\-\_\-elementow do 0. 

\begin{DoxyReturn}{Returns}
true jezeli funkcja jest pusta, w przeciwnym wypadku false. 
\end{DoxyReturn}


Definition at line 65 of file stos.\-cpp.

\hypertarget{classstos_a2ee24ebe7ef28d8fb0e37950931a548f}{\index{stos@{stos}!pop@{pop}}
\index{pop@{pop}!stos@{stos}}
\subsubsection[{pop}]{\setlength{\rightskip}{0pt plus 5cm}void stos\-::pop (
\begin{DoxyParamCaption}
\item[{int $\ast$}]{a}
\end{DoxyParamCaption}
)}}\label{classstos_a2ee24ebe7ef28d8fb0e37950931a548f}


Zdejmowanie elementu ze stosu. Funkcja zdejmuje element z konca stosu, jezeli liczba elementow bedzie mniejsza badz rowna 1/4 mozliwych elementow w tablicy, zostaje ona pomniejszana. 


\begin{DoxyParams}[1]{Parameters}
\mbox{\tt in}  & {\em a} & -\/ wskaznik do ktorego przypisywany jest element zdejmowany, aby mogl zostal uzyty w przyszlosci \\
\hline
\end{DoxyParams}
\begin{DoxyReturn}{Returns}
(brak) 
\end{DoxyReturn}


Definition at line 50 of file stos.\-cpp.

\hypertarget{classstos_ae3a9afe12a83444555ce0246233fa9e0}{\index{stos@{stos}!push@{push}}
\index{push@{push}!stos@{stos}}
\subsubsection[{push}]{\setlength{\rightskip}{0pt plus 5cm}void stos\-::push (
\begin{DoxyParamCaption}
\item[{int}]{el\-\_\-dodawany}
\end{DoxyParamCaption}
)}}\label{classstos_ae3a9afe12a83444555ce0246233fa9e0}


Dodanie elementu na stos. Funkcja dodaje element na koniec stosu, jezeli brakuje miejsca powieksza pamiec. 


\begin{DoxyParams}[1]{Parameters}
\mbox{\tt in}  & {\em el\-\_\-dodawany} & -\/ zmienna stala, ktora ma zostac dodana na koniec stosu \\
\hline
\end{DoxyParams}
\begin{DoxyReturn}{Returns}
(brak) 
\end{DoxyReturn}


Definition at line 41 of file stos.\-cpp.

\hypertarget{classstos_a47fcfe525e580ceb48080b33ab3d53de}{\index{stos@{stos}!size@{size}}
\index{size@{size}!stos@{stos}}
\subsubsection[{size}]{\setlength{\rightskip}{0pt plus 5cm}int stos\-::size (
\begin{DoxyParamCaption}
{}
\end{DoxyParamCaption}
)}}\label{classstos_a47fcfe525e580ceb48080b33ab3d53de}


Sprawdzenie rozmiaru funkcji Funkcja sprawdza ile elementow jest na stosie. 

\begin{DoxyReturn}{Returns}
Zwraca liczbe elementow znajdujacych sie na stosie 
\end{DoxyReturn}


Definition at line 69 of file stos.\-cpp.



The documentation for this class was generated from the following files\-:\begin{DoxyCompactItemize}
\item 
C\-:/\-Users/\-Ania/workspace/zadanie/inc/\hyperlink{stos_8hh}{stos.\-hh}\item 
C\-:/\-Users/\-Ania/workspace/zadanie/src/\hyperlink{stos_8cpp}{stos.\-cpp}\end{DoxyCompactItemize}

\hypertarget{classstoslista}{\section{stoslista Class Reference}
\label{classstoslista}\index{stoslista@{stoslista}}
}


Klasa modeluje pojecie stosu,bazujacego na liscie. Na stosie wyonywane sa operacje dodatnia na stos, zdjecia ze stosu, sprawdzenia czy jest pusty oraz sprawdzenia rozmiaru.  




{\ttfamily \#include $<$stos\-\_\-lista.\-hh$>$}

\subsection*{Public Member Functions}
\begin{DoxyCompactItemize}
\item 
void \hyperlink{classstoslista_aadf1d5447cbecd6ff9031732a7a8fa8d}{push} (int el\-\_\-dodawany)
\begin{DoxyCompactList}\small\item\em Dodanie elementu na stos. Funkcja dodaje element na koniec stosu. \end{DoxyCompactList}\item 
void \hyperlink{classstoslista_ab67ebb085a307bffa6c9dd2d6e341cc6}{pop} (int $\ast$a)
\begin{DoxyCompactList}\small\item\em Zdejmowanie elementu ze stosu. Funkcja zdejmuje element z konca stosu. \end{DoxyCompactList}\item 
bool \hyperlink{classstoslista_a723a7adfc6b429210d5b54bc126475e1}{isempty} ()
\begin{DoxyCompactList}\small\item\em Sprawdzenie czy funkcja jest pusta Funkcja sprawdza czy stos jest pusty poprzez uzycie funkcji size ktora dziala na liscie. \end{DoxyCompactList}\item 
int \hyperlink{classstoslista_adc6f1cf66d65fa6c6875ab38bd8ab4c6}{size} ()
\begin{DoxyCompactList}\small\item\em Zwrocenie rozmiaru funkcji Funkcja sprawdza ile elementow jest na stosie. \end{DoxyCompactList}\end{DoxyCompactItemize}


\subsection{Detailed Description}
Klasa modeluje pojecie stosu,bazujacego na liscie. Na stosie wyonywane sa operacje dodatnia na stos, zdjecia ze stosu, sprawdzenia czy jest pusty oraz sprawdzenia rozmiaru. 

Definition at line 25 of file stos\-\_\-lista.\-hh.



\subsection{Member Function Documentation}
\hypertarget{classstoslista_a723a7adfc6b429210d5b54bc126475e1}{\index{stoslista@{stoslista}!isempty@{isempty}}
\index{isempty@{isempty}!stoslista@{stoslista}}
\subsubsection[{isempty}]{\setlength{\rightskip}{0pt plus 5cm}bool stoslista\-::isempty (
\begin{DoxyParamCaption}
{}
\end{DoxyParamCaption}
)}}\label{classstoslista_a723a7adfc6b429210d5b54bc126475e1}


Sprawdzenie czy funkcja jest pusta Funkcja sprawdza czy stos jest pusty poprzez uzycie funkcji size ktora dziala na liscie. 

\begin{DoxyReturn}{Returns}
true jezeli funkcja jest pusta, w przeciwnym wypadku false. 
\end{DoxyReturn}


Definition at line 29 of file stos\-\_\-lista.\-cpp.

\hypertarget{classstoslista_ab67ebb085a307bffa6c9dd2d6e341cc6}{\index{stoslista@{stoslista}!pop@{pop}}
\index{pop@{pop}!stoslista@{stoslista}}
\subsubsection[{pop}]{\setlength{\rightskip}{0pt plus 5cm}void stoslista\-::pop (
\begin{DoxyParamCaption}
\item[{int $\ast$}]{a}
\end{DoxyParamCaption}
)}}\label{classstoslista_ab67ebb085a307bffa6c9dd2d6e341cc6}


Zdejmowanie elementu ze stosu. Funkcja zdejmuje element z konca stosu. 


\begin{DoxyParams}[1]{Parameters}
\mbox{\tt in}  & {\em a} & -\/ wskaznik do ktorego przypisywany jest element zdejmowany, aby mogl zostal uzyty w przyszlosci \\
\hline
\end{DoxyParams}
\begin{DoxyReturn}{Returns}
(brak) 
\end{DoxyReturn}


Definition at line 17 of file stos\-\_\-lista.\-cpp.

\hypertarget{classstoslista_aadf1d5447cbecd6ff9031732a7a8fa8d}{\index{stoslista@{stoslista}!push@{push}}
\index{push@{push}!stoslista@{stoslista}}
\subsubsection[{push}]{\setlength{\rightskip}{0pt plus 5cm}void stoslista\-::push (
\begin{DoxyParamCaption}
\item[{int}]{el\-\_\-dodawany}
\end{DoxyParamCaption}
)}}\label{classstoslista_aadf1d5447cbecd6ff9031732a7a8fa8d}


Dodanie elementu na stos. Funkcja dodaje element na koniec stosu. 


\begin{DoxyParams}[1]{Parameters}
\mbox{\tt in}  & {\em el\-\_\-dodawany} & -\/ zmienna stala, ktora ma zostac dodana na koniec stosu \\
\hline
\end{DoxyParams}
\begin{DoxyReturn}{Returns}
(brak) 
\end{DoxyReturn}


Definition at line 12 of file stos\-\_\-lista.\-cpp.

\hypertarget{classstoslista_adc6f1cf66d65fa6c6875ab38bd8ab4c6}{\index{stoslista@{stoslista}!size@{size}}
\index{size@{size}!stoslista@{stoslista}}
\subsubsection[{size}]{\setlength{\rightskip}{0pt plus 5cm}int stoslista\-::size (
\begin{DoxyParamCaption}
{}
\end{DoxyParamCaption}
)}}\label{classstoslista_adc6f1cf66d65fa6c6875ab38bd8ab4c6}


Zwrocenie rozmiaru funkcji Funkcja sprawdza ile elementow jest na stosie. 

\begin{DoxyReturn}{Returns}
Zwraca liczbe elementow znajdujacych sie na stosie. 
\end{DoxyReturn}


Definition at line 33 of file stos\-\_\-lista.\-cpp.



The documentation for this class was generated from the following files\-:\begin{DoxyCompactItemize}
\item 
C\-:/\-Users/\-Ania/workspace/zadanie/inc/\hyperlink{stos__lista_8hh}{stos\-\_\-lista.\-hh}\item 
C\-:/\-Users/\-Ania/workspace/zadanie/src/\hyperlink{stos__lista_8cpp}{stos\-\_\-lista.\-cpp}\end{DoxyCompactItemize}

\hypertarget{classzegar}{\section{zegar Class Reference}
\label{classzegar}\index{zegar@{zegar}}
}


Klasa modeluje uruchomienia g�ownych wlasciowosci programu. Atrybutem klasy sa stowrzone dwa elemnty klasy dane, na ktorych wykonywane sa dzialania.  




{\ttfamily \#include $<$uruchom.\-hh$>$}

\subsection*{Public Member Functions}
\begin{DoxyCompactItemize}
\item 
\hyperlink{classzegar_a5a056093f44cf4af3e113c9120c4fe89}{zegar} ()
\item 
void \hyperlink{classzegar_a33ed36b856cb3e963cf237112190b088}{wczytaj\-\_\-dane\-\_\-pod} (string nazwa\-\_\-pliku\-\_\-pod)
\begin{DoxyCompactList}\small\item\em Wczytanie danych podstawowych Funkcja wczytuje dane, na ktorych wykonywany jest algorytm. Dane te sa glowna funkcja programu. \end{DoxyCompactList}\item 
void \hyperlink{classzegar_abd42eadcb229d63bfd5294168b56bbe7}{wczytaj\-\_\-dane\-\_\-spr} (string nazwa\-\_\-pliku\-\_\-spr)
\begin{DoxyCompactList}\small\item\em Wczytanie danych sprawdzajacych Funkcja wczytuje dane, na ktore zostaja porwonane z danymi na ktorych zostal wykonany algorytm. \end{DoxyCompactList}\item 
void \hyperlink{classzegar_a8975a0605f6d9a6c8556459ec429e1a4}{algorytm} ()
\begin{DoxyCompactList}\small\item\em Funkcja wykonuj�ca zadany algorytm na wektorze. Funkcja wykonuje zadany algorytm na wektorze wejsciowym. W naszym przypadku wektor pomno�ony jest przez sta�� liczb� 2. \end{DoxyCompactList}\item 
bool \hyperlink{classzegar_ab62a10bb1731935c342405ce39580b66}{porownaj} ()
\begin{DoxyCompactList}\small\item\em Funkcja por�wnuj�ca dwa wektory. Funkcja por�wnuje dwa wektory, sprawdza czy wszystkie elemnty s� ze sob� r�wne. \end{DoxyCompactList}\item 
L\-A\-R\-G\-E\-\_\-\-I\-N\-T\-E\-G\-E\-R \hyperlink{classzegar_aeea0981af9bbff5af057ca2462d2f574}{wlacz\-Stoper} ()
\begin{DoxyCompactList}\small\item\em Funkcja zapamietuj�ca czas poczatkowy. Funkcja nale��ca do biblioteki \char`\"{}windows.\-h\char`\"{}, stoper zostaje w��czony. Funkcja nale��ca do funkcji bool Query\-Performance\-Counter(\-\_\-out L\-A\-R\-G\-E\-\_\-\-I\-N\-T\-E\-G\-E\-R $\ast$\-Ip\-Performance\-Count), funkcja ta zwraca wartosc niezerowa je�eli w�aczenie zako�czy si� sukcesem, natomiast w przeciwnym wypadku zostanie wyrzucony b��d i zwr�ci wartosc 0. Dla komputer�w multiprocesorowych nie ma znaczenia, kt�ry jest u�ywany, mog� jedynie r�ni� si� minimalnie czasy. \end{DoxyCompactList}\item 
L\-A\-R\-G\-E\-\_\-\-I\-N\-T\-E\-G\-E\-R \hyperlink{classzegar_af1d85ecece1c8e7c08f32a761f383b2c}{wylacz\-Stoper} ()
\begin{DoxyCompactList}\small\item\em Funkcja zapamietuj�ca czas ko�cowy. Funkcja nale��ca do biblioteki \char`\"{}windows.\-h\char`\"{}, stoper zostaje wy��czony, aby zosta� zmierzony czas wykonania algorytmu w programie, poprzez odj�cie czasu pocz�tkowego od czasu ko�cowego. Funkcja nale��ca do funkcji bool Query\-Performance\-Counter(\-\_\-out L\-A\-R\-G\-E\-\_\-\-I\-N\-T\-E\-G\-E\-R $\ast$\-Ip\-Performance\-Count), funkcja ta zwraca wartosc niezerowa je�eli wy�aczenie zako�czy si� sukcesem, natomiast w przeciwnym wypadku zostanie wyrzucony b��d i zwr�ci wartosc 0. Dla komputer�w multiprocesorowych nie ma znaczenia, kt�ry jest u�ywany, mog� jedynie r�ni� si� minimalnie czasy. \end{DoxyCompactList}\item 
void \hyperlink{classzegar_a4b09c651606f678bacda917f08acee2e}{wczytaj} (string nazwa)
\begin{DoxyCompactList}\small\item\em Funkcja wczytuj�ca dane do stosu. Funkcja otwiera plik zdefiniowany w glownej funkcji przez u�ytkownika, sprawdza czy plik zosta� otwarty, je�eli zosta� otwarty wczytana jest liczba elementow pliku, nastepnie wczytywane sa wszystkie liczby do tablicy. \end{DoxyCompactList}\end{DoxyCompactItemize}


\subsection{Detailed Description}
Klasa modeluje uruchomienia g�ownych wlasciowosci programu. Atrybutem klasy sa stowrzone dwa elemnty klasy dane, na ktorych wykonywane sa dzialania. 

Definition at line 28 of file uruchom.\-hh.



\subsection{Constructor \& Destructor Documentation}
\hypertarget{classzegar_a5a056093f44cf4af3e113c9120c4fe89}{\index{zegar@{zegar}!zegar@{zegar}}
\index{zegar@{zegar}!zegar@{zegar}}
\subsubsection[{zegar}]{\setlength{\rightskip}{0pt plus 5cm}zegar\-::zegar (
\begin{DoxyParamCaption}
{}
\end{DoxyParamCaption}
)\hspace{0.3cm}{\ttfamily [inline]}}}\label{classzegar_a5a056093f44cf4af3e113c9120c4fe89}


Definition at line 45 of file uruchom.\-hh.



\subsection{Member Function Documentation}
\hypertarget{classzegar_a8975a0605f6d9a6c8556459ec429e1a4}{\index{zegar@{zegar}!algorytm@{algorytm}}
\index{algorytm@{algorytm}!zegar@{zegar}}
\subsubsection[{algorytm}]{\setlength{\rightskip}{0pt plus 5cm}void zegar\-::algorytm (
\begin{DoxyParamCaption}
{}
\end{DoxyParamCaption}
)}}\label{classzegar_a8975a0605f6d9a6c8556459ec429e1a4}


Funkcja wykonuj�ca zadany algorytm na wektorze. Funkcja wykonuje zadany algorytm na wektorze wejsciowym. W naszym przypadku wektor pomno�ony jest przez sta�� liczb� 2. 

\begin{DoxyReturn}{Returns}
(brak) 
\end{DoxyReturn}


Definition at line 14 of file uruchom.\-cpp.

\hypertarget{classzegar_ab62a10bb1731935c342405ce39580b66}{\index{zegar@{zegar}!porownaj@{porownaj}}
\index{porownaj@{porownaj}!zegar@{zegar}}
\subsubsection[{porownaj}]{\setlength{\rightskip}{0pt plus 5cm}bool zegar\-::porownaj (
\begin{DoxyParamCaption}
{}
\end{DoxyParamCaption}
)\hspace{0.3cm}{\ttfamily [inline]}}}\label{classzegar_ab62a10bb1731935c342405ce39580b66}


Funkcja por�wnuj�ca dwa wektory. Funkcja por�wnuje dwa wektory, sprawdza czy wszystkie elemnty s� ze sob� r�wne. 

\begin{DoxyReturn}{Returns}
Funkcja zwraca true jezeli wektory sa jednakowe w przeciwnym wypadku zostaje wzrocony false. 
\end{DoxyReturn}


Definition at line 83 of file uruchom.\-hh.

\hypertarget{classzegar_a4b09c651606f678bacda917f08acee2e}{\index{zegar@{zegar}!wczytaj@{wczytaj}}
\index{wczytaj@{wczytaj}!zegar@{zegar}}
\subsubsection[{wczytaj}]{\setlength{\rightskip}{0pt plus 5cm}void zegar\-::wczytaj (
\begin{DoxyParamCaption}
\item[{string}]{nazwa}
\end{DoxyParamCaption}
)}}\label{classzegar_a4b09c651606f678bacda917f08acee2e}


Funkcja wczytuj�ca dane do stosu. Funkcja otwiera plik zdefiniowany w glownej funkcji przez u�ytkownika, sprawdza czy plik zosta� otwarty, je�eli zosta� otwarty wczytana jest liczba elementow pliku, nastepnie wczytywane sa wszystkie liczby do tablicy. 


\begin{DoxyParams}[1]{Parameters}
\mbox{\tt in}  & {\em nazwa} & -\/ zmienna, kt�ra zostaje wprowadzona przez u�ytkownika do programu \\
\hline
\end{DoxyParams}
\begin{DoxyReturn}{Returns}
(brak) 
\end{DoxyReturn}


Definition at line 38 of file uruchom.\-cpp.

\hypertarget{classzegar_a33ed36b856cb3e963cf237112190b088}{\index{zegar@{zegar}!wczytaj\-\_\-dane\-\_\-pod@{wczytaj\-\_\-dane\-\_\-pod}}
\index{wczytaj\-\_\-dane\-\_\-pod@{wczytaj\-\_\-dane\-\_\-pod}!zegar@{zegar}}
\subsubsection[{wczytaj\-\_\-dane\-\_\-pod}]{\setlength{\rightskip}{0pt plus 5cm}void zegar\-::wczytaj\-\_\-dane\-\_\-pod (
\begin{DoxyParamCaption}
\item[{string}]{nazwa\-\_\-pliku\-\_\-pod}
\end{DoxyParamCaption}
)\hspace{0.3cm}{\ttfamily [inline]}}}\label{classzegar_a33ed36b856cb3e963cf237112190b088}


Wczytanie danych podstawowych Funkcja wczytuje dane, na ktorych wykonywany jest algorytm. Dane te sa glowna funkcja programu. 


\begin{DoxyParams}[1]{Parameters}
\mbox{\tt in}  & {\em nazwa\-\_\-pliku\-\_\-pod} & -\/ jest to zmienna ktora jest ciagiem znakow (nazwa pliku), ktory ma zostac otwarty. \\
\hline
\end{DoxyParams}
\begin{DoxyReturn}{Returns}
(brak) 
\end{DoxyReturn}


Definition at line 57 of file uruchom.\-hh.

\hypertarget{classzegar_abd42eadcb229d63bfd5294168b56bbe7}{\index{zegar@{zegar}!wczytaj\-\_\-dane\-\_\-spr@{wczytaj\-\_\-dane\-\_\-spr}}
\index{wczytaj\-\_\-dane\-\_\-spr@{wczytaj\-\_\-dane\-\_\-spr}!zegar@{zegar}}
\subsubsection[{wczytaj\-\_\-dane\-\_\-spr}]{\setlength{\rightskip}{0pt plus 5cm}void zegar\-::wczytaj\-\_\-dane\-\_\-spr (
\begin{DoxyParamCaption}
\item[{string}]{nazwa\-\_\-pliku\-\_\-spr}
\end{DoxyParamCaption}
)\hspace{0.3cm}{\ttfamily [inline]}}}\label{classzegar_abd42eadcb229d63bfd5294168b56bbe7}


Wczytanie danych sprawdzajacych Funkcja wczytuje dane, na ktore zostaja porwonane z danymi na ktorych zostal wykonany algorytm. 


\begin{DoxyParams}[1]{Parameters}
\mbox{\tt in}  & {\em nazwa\-\_\-pliku\-\_\-spr} & -\/ jest to zmienna ktora jest ciagiem znakow (nazwa pliku), ktory ma zostac otwarty, w celu porownania. \\
\hline
\end{DoxyParams}
\begin{DoxyReturn}{Returns}
(brak) 
\end{DoxyReturn}


Definition at line 68 of file uruchom.\-hh.

\hypertarget{classzegar_aeea0981af9bbff5af057ca2462d2f574}{\index{zegar@{zegar}!wlacz\-Stoper@{wlacz\-Stoper}}
\index{wlacz\-Stoper@{wlacz\-Stoper}!zegar@{zegar}}
\subsubsection[{wlacz\-Stoper}]{\setlength{\rightskip}{0pt plus 5cm}L\-A\-R\-G\-E\-\_\-\-I\-N\-T\-E\-G\-E\-R zegar\-::wlacz\-Stoper (
\begin{DoxyParamCaption}
{}
\end{DoxyParamCaption}
)}}\label{classzegar_aeea0981af9bbff5af057ca2462d2f574}


Funkcja zapamietuj�ca czas poczatkowy. Funkcja nale��ca do biblioteki \char`\"{}windows.\-h\char`\"{}, stoper zostaje w��czony. Funkcja nale��ca do funkcji bool Query\-Performance\-Counter(\-\_\-out L\-A\-R\-G\-E\-\_\-\-I\-N\-T\-E\-G\-E\-R $\ast$\-Ip\-Performance\-Count), funkcja ta zwraca wartosc niezerowa je�eli w�aczenie zako�czy si� sukcesem, natomiast w przeciwnym wypadku zostanie wyrzucony b��d i zwr�ci wartosc 0. Dla komputer�w multiprocesorowych nie ma znaczenia, kt�ry jest u�ywany, mog� jedynie r�ni� si� minimalnie czasy. 



Definition at line 20 of file uruchom.\-cpp.

\hypertarget{classzegar_af1d85ecece1c8e7c08f32a761f383b2c}{\index{zegar@{zegar}!wylacz\-Stoper@{wylacz\-Stoper}}
\index{wylacz\-Stoper@{wylacz\-Stoper}!zegar@{zegar}}
\subsubsection[{wylacz\-Stoper}]{\setlength{\rightskip}{0pt plus 5cm}L\-A\-R\-G\-E\-\_\-\-I\-N\-T\-E\-G\-E\-R zegar\-::wylacz\-Stoper (
\begin{DoxyParamCaption}
{}
\end{DoxyParamCaption}
)}}\label{classzegar_af1d85ecece1c8e7c08f32a761f383b2c}


Funkcja zapamietuj�ca czas ko�cowy. Funkcja nale��ca do biblioteki \char`\"{}windows.\-h\char`\"{}, stoper zostaje wy��czony, aby zosta� zmierzony czas wykonania algorytmu w programie, poprzez odj�cie czasu pocz�tkowego od czasu ko�cowego. Funkcja nale��ca do funkcji bool Query\-Performance\-Counter(\-\_\-out L\-A\-R\-G\-E\-\_\-\-I\-N\-T\-E\-G\-E\-R $\ast$\-Ip\-Performance\-Count), funkcja ta zwraca wartosc niezerowa je�eli wy�aczenie zako�czy si� sukcesem, natomiast w przeciwnym wypadku zostanie wyrzucony b��d i zwr�ci wartosc 0. Dla komputer�w multiprocesorowych nie ma znaczenia, kt�ry jest u�ywany, mog� jedynie r�ni� si� minimalnie czasy. 



Definition at line 29 of file uruchom.\-cpp.



The documentation for this class was generated from the following files\-:\begin{DoxyCompactItemize}
\item 
C\-:/\-Users/\-Ania/workspace/zadanie/inc/\hyperlink{uruchom_8hh}{uruchom.\-hh}\item 
C\-:/\-Users/\-Ania/workspace/zadanie/src/\hyperlink{uruchom_8cpp}{uruchom.\-cpp}\end{DoxyCompactItemize}

\chapter{File Documentation}
\hypertarget{strona_8dox}{\section{C\-:/\-Users/\-Ania/workspace/zadanie/doc/pages/strona.dox File Reference}
\label{strona_8dox}\index{C\-:/\-Users/\-Ania/workspace/zadanie/doc/pages/strona.\-dox@{C\-:/\-Users/\-Ania/workspace/zadanie/doc/pages/strona.\-dox}}
}

\hypertarget{dane_8hh}{\section{C\-:/\-Users/\-Ania/workspace/zadanie/inc/dane.hh File Reference}
\label{dane_8hh}\index{C\-:/\-Users/\-Ania/workspace/zadanie/inc/dane.\-hh@{C\-:/\-Users/\-Ania/workspace/zadanie/inc/dane.\-hh}}
}


Definicja klasy dane.  


{\ttfamily \#include $<$iostream$>$}\\*
{\ttfamily \#include $<$string$>$}\\*
{\ttfamily \#include $<$vector$>$}\\*
\subsection*{Classes}
\begin{DoxyCompactItemize}
\item 
class \hyperlink{classdane}{dane}
\begin{DoxyCompactList}\small\item\em Modeluje pojecie danych, u�ytych w programie, kt�re w moim przypadku sa wektorem, ktory zostaje wczytany z pliku. Pierwsza zmienna wczytana z pliku jest liczba elemntow wystepujacych w tym pliku. Przyjelam, ze zmienne wczytane z pliku sa liczbami calkowitymi (int). \end{DoxyCompactList}\end{DoxyCompactItemize}


\subsection{Detailed Description}
Definicja klasy dane. Plik zawiera definicje klasy dane, ktora jest klasa podstawowa programu 

Definition in file \hyperlink{dane_8hh_source}{dane.\-hh}.


\hypertarget{kolejka_8hh}{\section{C\-:/\-Users/\-Ania/workspace/zadanie/inc/kolejka.hh File Reference}
\label{kolejka_8hh}\index{C\-:/\-Users/\-Ania/workspace/zadanie/inc/kolejka.\-hh@{C\-:/\-Users/\-Ania/workspace/zadanie/inc/kolejka.\-hh}}
}


Definicja klasy kolejkatab.  


{\ttfamily \#include \char`\"{}uruchom.\-hh\char`\"{}}\\*
\subsection*{Classes}
\begin{DoxyCompactItemize}
\item 
class \hyperlink{classkolejkatab}{kolejkatab}
\begin{DoxyCompactList}\small\item\em Klasa modeluje pojecie kolejki, bazujacej na tablicy. Wykonywane sa operacje dodawnia do kolejki, zdjecia z kolejki, sprawdzenia czy jest pusta oraz sprawdzenia rozmiaru. Tablice zaalokowalam w sposob dynamiczny. Dodatkowo u�yte niektore funkcje pomocnicze jak np. powiekszenie tablicy. \end{DoxyCompactList}\end{DoxyCompactItemize}


\subsection{Detailed Description}
Definicja klasy kolejkatab. Plik zawiera definicje klasy kolejkatab, w ktorej wykorzystana jest tablica do zapisania elementow w kolejce. 

Definition in file \hyperlink{kolejka_8hh_source}{kolejka.\-hh}.


\hypertarget{kolejka__lista_8hh}{\section{C\-:/\-Users/\-Ania/workspace/zadanie/inc/kolejka\-\_\-lista.hh File Reference}
\label{kolejka__lista_8hh}\index{C\-:/\-Users/\-Ania/workspace/zadanie/inc/kolejka\-\_\-lista.\-hh@{C\-:/\-Users/\-Ania/workspace/zadanie/inc/kolejka\-\_\-lista.\-hh}}
}


Definicja klasy kolejkalista.  


{\ttfamily \#include $<$list$>$}\\*
{\ttfamily \#include $<$iostream$>$}\\*
{\ttfamily \#include $<$cstdlib$>$}\\*
\subsection*{Classes}
\begin{DoxyCompactItemize}
\item 
class \hyperlink{classkolejkalista}{kolejkalista}
\begin{DoxyCompactList}\small\item\em Klasa modeluje pojecie kolejki, bazujacej na liscie. Wykonywane sa operacje dodawnia fo kolejki, zdjecia z kolejki, sprawdzenia czy jest pusta oraz sprawdzenia rozmiaru. Tablice zaalokowalam w sposob dynamiczny. \end{DoxyCompactList}\end{DoxyCompactItemize}


\subsection{Detailed Description}
Definicja klasy kolejkalista. Plik zawiera definicje klasy kolejkalista, w ktorej wykorzystana jest lista do zapisania elementow w kolejce. 

Definition in file \hyperlink{kolejka__lista_8hh_source}{kolejka\-\_\-lista.\-hh}.


\hypertarget{stos_8hh}{\section{C\-:/\-Users/\-Ania/workspace/zadanie/inc/stos.hh File Reference}
\label{stos_8hh}\index{C\-:/\-Users/\-Ania/workspace/zadanie/inc/stos.\-hh@{C\-:/\-Users/\-Ania/workspace/zadanie/inc/stos.\-hh}}
}


Definicja klasy stos.  


{\ttfamily \#include $<$iostream$>$}\\*
{\ttfamily \#include $<$cstdlib$>$}\\*
\subsection*{Classes}
\begin{DoxyCompactItemize}
\item 
class \hyperlink{classstos}{stos}
\begin{DoxyCompactList}\small\item\em Klasa modeluje pojecie stosu,bazujacego na tablicy. Na stosie wyonywane sa operacje dodatnia na stos, zdjecia ze stosu, sprawdzenia czy jest pusty oraz sprawdzenia rozmiaru. Tablice zaalokowalam w sposob dynamiczny. Dodatkowo u�yte niektore funkcje pomocnicze jak np. powiekszenie stosu. \end{DoxyCompactList}\end{DoxyCompactItemize}


\subsection{Detailed Description}
Definicja klasy stos. Plik zawiera definicje klasy stos, ktora do zapisu i zapamietania liczb uzywa tablice. 

Definition in file \hyperlink{stos_8hh_source}{stos.\-hh}.


\hypertarget{stos__lista_8hh}{\section{C\-:/\-Users/\-Ania/workspace/zadanie/inc/stos\-\_\-lista.hh File Reference}
\label{stos__lista_8hh}\index{C\-:/\-Users/\-Ania/workspace/zadanie/inc/stos\-\_\-lista.\-hh@{C\-:/\-Users/\-Ania/workspace/zadanie/inc/stos\-\_\-lista.\-hh}}
}


Definicja klasy stoslista.  


{\ttfamily \#include $<$list$>$}\\*
{\ttfamily \#include $<$iostream$>$}\\*
{\ttfamily \#include $<$cstdlib$>$}\\*
\subsection*{Classes}
\begin{DoxyCompactItemize}
\item 
class \hyperlink{classstoslista}{stoslista}
\begin{DoxyCompactList}\small\item\em Klasa modeluje pojecie stosu,bazujacego na liscie. Na stosie wyonywane sa operacje dodatnia na stos, zdjecia ze stosu, sprawdzenia czy jest pusty oraz sprawdzenia rozmiaru. \end{DoxyCompactList}\end{DoxyCompactItemize}


\subsection{Detailed Description}
Definicja klasy stoslista. Plik zawiera definicje klasy stoslista, ktora do zapisu i zapamietania liczb uzywa listy. 

Definition in file \hyperlink{stos__lista_8hh_source}{stos\-\_\-lista.\-hh}.


\hypertarget{uruchom_8hh}{\section{C\-:/\-Users/\-Ania/workspace/zadanie/inc/uruchom.hh File Reference}
\label{uruchom_8hh}\index{C\-:/\-Users/\-Ania/workspace/zadanie/inc/uruchom.\-hh@{C\-:/\-Users/\-Ania/workspace/zadanie/inc/uruchom.\-hh}}
}


Definicja klasy zegar.  


{\ttfamily \#include \char`\"{}dane.\-hh\char`\"{}}\\*
{\ttfamily \#include $<$windows.\-h$>$}\\*
{\ttfamily \#include \char`\"{}stos.\-hh\char`\"{}}\\*
\subsection*{Classes}
\begin{DoxyCompactItemize}
\item 
class \hyperlink{classzegar}{zegar}
\begin{DoxyCompactList}\small\item\em Klasa modeluje uruchomienia g�ownych wlasciowosci programu. Atrybutem klasy sa stowrzone dwa elemnty klasy dane, na ktorych wykonywane sa dzialania. \end{DoxyCompactList}\end{DoxyCompactItemize}


\subsection{Detailed Description}
Definicja klasy zegar. Plik zawiera definicje klasy zegar, ktora jest klasa glowna programu. Klasa ta jest pochodna i specjalizacja klasy dane. 

Definition in file \hyperlink{uruchom_8hh_source}{uruchom.\-hh}.


\hypertarget{dane_8cpp}{\section{C\-:/\-Users/\-Ania/workspace/zadanie/src/dane.cpp File Reference}
\label{dane_8cpp}\index{C\-:/\-Users/\-Ania/workspace/zadanie/src/dane.\-cpp@{C\-:/\-Users/\-Ania/workspace/zadanie/src/dane.\-cpp}}
}
{\ttfamily \#include \char`\"{}dane.\-hh\char`\"{}}\\*
{\ttfamily \#include $<$fstream$>$}\\*

\hypertarget{kolejka_8cpp}{\section{C\-:/\-Users/\-Ania/workspace/zadanie/src/kolejka.cpp File Reference}
\label{kolejka_8cpp}\index{C\-:/\-Users/\-Ania/workspace/zadanie/src/kolejka.\-cpp@{C\-:/\-Users/\-Ania/workspace/zadanie/src/kolejka.\-cpp}}
}
{\ttfamily \#include \char`\"{}kolejka.\-hh\char`\"{}}\\*

\hypertarget{kolejka__lista_8cpp}{\section{C\-:/\-Users/\-Ania/workspace/zadanie/src/kolejka\-\_\-lista.cpp File Reference}
\label{kolejka__lista_8cpp}\index{C\-:/\-Users/\-Ania/workspace/zadanie/src/kolejka\-\_\-lista.\-cpp@{C\-:/\-Users/\-Ania/workspace/zadanie/src/kolejka\-\_\-lista.\-cpp}}
}
{\ttfamily \#include \char`\"{}kolejka\-\_\-lista.\-hh\char`\"{}}\\*

\hypertarget{main_8cpp}{\section{C\-:/\-Users/\-Ania/workspace/zadanie/src/main.cpp File Reference}
\label{main_8cpp}\index{C\-:/\-Users/\-Ania/workspace/zadanie/src/main.\-cpp@{C\-:/\-Users/\-Ania/workspace/zadanie/src/main.\-cpp}}
}


Funkcja glowna ktorej glownym zalozeniem jest wczytanie plikow z rozna wielkoscia elementow znajdujacych si� w pliku, obliczenie sredniej wartosci czasu, w jakim zostaje wykonany algorytm (w naszym przypadku pomnozenie przez 2),nastepnie program porownuje poprawnosc wykoannia mnozenia z plikiem sprawdzajacym. Uzytkownik musi w programie zdefiniowac\-: liczbe powtorzen (zmienna j), ilosc plikow -\/ do ilu wykonywane jest mnozenie (zmienna i), nazwy plikow (string czesc\-\_\-1, i, czesc\-\_\-2 -\/ wszystko opcjonalnie).  


{\ttfamily \#include \char`\"{}uruchom.\-hh\char`\"{}}\\*
{\ttfamily \#include $<$sstream$>$}\\*
{\ttfamily \#include $<$fstream$>$}\\*
\subsection*{Functions}
\begin{DoxyCompactItemize}
\item 
int \hyperlink{main_8cpp_ae66f6b31b5ad750f1fe042a706a4e3d4}{main} ()
\end{DoxyCompactItemize}


\subsection{Detailed Description}
Funkcja glowna ktorej glownym zalozeniem jest wczytanie plikow z rozna wielkoscia elementow znajdujacych si� w pliku, obliczenie sredniej wartosci czasu, w jakim zostaje wykonany algorytm (w naszym przypadku pomnozenie przez 2),nastepnie program porownuje poprawnosc wykoannia mnozenia z plikiem sprawdzajacym. Uzytkownik musi w programie zdefiniowac\-: liczbe powtorzen (zmienna j), ilosc plikow -\/ do ilu wykonywane jest mnozenie (zmienna i), nazwy plikow (string czesc\-\_\-1, i, czesc\-\_\-2 -\/ wszystko opcjonalnie). \begin{DoxyReturn}{Returns}
(brak) 
\end{DoxyReturn}


Definition in file \hyperlink{main_8cpp_source}{main.\-cpp}.



\subsection{Function Documentation}
\hypertarget{main_8cpp_ae66f6b31b5ad750f1fe042a706a4e3d4}{\index{main.\-cpp@{main.\-cpp}!main@{main}}
\index{main@{main}!main.cpp@{main.\-cpp}}
\subsubsection[{main}]{\setlength{\rightskip}{0pt plus 5cm}int main (
\begin{DoxyParamCaption}
{}
\end{DoxyParamCaption}
)}}\label{main_8cpp_ae66f6b31b5ad750f1fe042a706a4e3d4}


Definition at line 19 of file main.\-cpp.


\hypertarget{stos_8cpp}{\section{C\-:/\-Users/\-Ania/workspace/zadanie/src/stos.cpp File Reference}
\label{stos_8cpp}\index{C\-:/\-Users/\-Ania/workspace/zadanie/src/stos.\-cpp@{C\-:/\-Users/\-Ania/workspace/zadanie/src/stos.\-cpp}}
}
{\ttfamily \#include \char`\"{}stos.\-hh\char`\"{}}\\*

\hypertarget{stos__lista_8cpp}{\section{C\-:/\-Users/\-Ania/workspace/zadanie/src/stos\-\_\-lista.cpp File Reference}
\label{stos__lista_8cpp}\index{C\-:/\-Users/\-Ania/workspace/zadanie/src/stos\-\_\-lista.\-cpp@{C\-:/\-Users/\-Ania/workspace/zadanie/src/stos\-\_\-lista.\-cpp}}
}
{\ttfamily \#include \char`\"{}stos\-\_\-lista.\-hh\char`\"{}}\\*

\hypertarget{uruchom_8cpp}{\section{C\-:/\-Users/\-Ania/workspace/zadanie/src/uruchom.cpp File Reference}
\label{uruchom_8cpp}\index{C\-:/\-Users/\-Ania/workspace/zadanie/src/uruchom.\-cpp@{C\-:/\-Users/\-Ania/workspace/zadanie/src/uruchom.\-cpp}}
}
{\ttfamily \#include \char`\"{}uruchom.\-hh\char`\"{}}\\*
{\ttfamily \#include $<$fstream$>$}\\*

\printindex
\end{document}
